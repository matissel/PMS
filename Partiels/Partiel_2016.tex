\documentclass[]{article}
\usepackage{lmodern}
\usepackage{amssymb,amsmath}
\usepackage{ifxetex,ifluatex}
\usepackage{fixltx2e} % provides \textsubscript
\ifnum 0\ifxetex 1\fi\ifluatex 1\fi=0 % if pdftex
  \usepackage[T1]{fontenc}
  \usepackage[utf8]{inputenc}
\else % if luatex or xelatex
  \ifxetex
    \usepackage{mathspec}
  \else
    \usepackage{fontspec}
  \fi
  \defaultfontfeatures{Ligatures=TeX,Scale=MatchLowercase}
\fi
% use upquote if available, for straight quotes in verbatim environments
\IfFileExists{upquote.sty}{\usepackage{upquote}}{}
% use microtype if available
\IfFileExists{microtype.sty}{%
\usepackage{microtype}
\UseMicrotypeSet[protrusion]{basicmath} % disable protrusion for tt fonts
}{}
\usepackage[margin=1in]{geometry}
\usepackage{hyperref}
\hypersetup{unicode=true,
            pdftitle={Partiel 2016},
            pdfauthor={Matisse Landais},
            pdfborder={0 0 0},
            breaklinks=true}
\urlstyle{same}  % don't use monospace font for urls
\usepackage{color}
\usepackage{fancyvrb}
\newcommand{\VerbBar}{|}
\newcommand{\VERB}{\Verb[commandchars=\\\{\}]}
\DefineVerbatimEnvironment{Highlighting}{Verbatim}{commandchars=\\\{\}}
% Add ',fontsize=\small' for more characters per line
\usepackage{framed}
\definecolor{shadecolor}{RGB}{248,248,248}
\newenvironment{Shaded}{\begin{snugshade}}{\end{snugshade}}
\newcommand{\KeywordTok}[1]{\textcolor[rgb]{0.13,0.29,0.53}{\textbf{#1}}}
\newcommand{\DataTypeTok}[1]{\textcolor[rgb]{0.13,0.29,0.53}{#1}}
\newcommand{\DecValTok}[1]{\textcolor[rgb]{0.00,0.00,0.81}{#1}}
\newcommand{\BaseNTok}[1]{\textcolor[rgb]{0.00,0.00,0.81}{#1}}
\newcommand{\FloatTok}[1]{\textcolor[rgb]{0.00,0.00,0.81}{#1}}
\newcommand{\ConstantTok}[1]{\textcolor[rgb]{0.00,0.00,0.00}{#1}}
\newcommand{\CharTok}[1]{\textcolor[rgb]{0.31,0.60,0.02}{#1}}
\newcommand{\SpecialCharTok}[1]{\textcolor[rgb]{0.00,0.00,0.00}{#1}}
\newcommand{\StringTok}[1]{\textcolor[rgb]{0.31,0.60,0.02}{#1}}
\newcommand{\VerbatimStringTok}[1]{\textcolor[rgb]{0.31,0.60,0.02}{#1}}
\newcommand{\SpecialStringTok}[1]{\textcolor[rgb]{0.31,0.60,0.02}{#1}}
\newcommand{\ImportTok}[1]{#1}
\newcommand{\CommentTok}[1]{\textcolor[rgb]{0.56,0.35,0.01}{\textit{#1}}}
\newcommand{\DocumentationTok}[1]{\textcolor[rgb]{0.56,0.35,0.01}{\textbf{\textit{#1}}}}
\newcommand{\AnnotationTok}[1]{\textcolor[rgb]{0.56,0.35,0.01}{\textbf{\textit{#1}}}}
\newcommand{\CommentVarTok}[1]{\textcolor[rgb]{0.56,0.35,0.01}{\textbf{\textit{#1}}}}
\newcommand{\OtherTok}[1]{\textcolor[rgb]{0.56,0.35,0.01}{#1}}
\newcommand{\FunctionTok}[1]{\textcolor[rgb]{0.00,0.00,0.00}{#1}}
\newcommand{\VariableTok}[1]{\textcolor[rgb]{0.00,0.00,0.00}{#1}}
\newcommand{\ControlFlowTok}[1]{\textcolor[rgb]{0.13,0.29,0.53}{\textbf{#1}}}
\newcommand{\OperatorTok}[1]{\textcolor[rgb]{0.81,0.36,0.00}{\textbf{#1}}}
\newcommand{\BuiltInTok}[1]{#1}
\newcommand{\ExtensionTok}[1]{#1}
\newcommand{\PreprocessorTok}[1]{\textcolor[rgb]{0.56,0.35,0.01}{\textit{#1}}}
\newcommand{\AttributeTok}[1]{\textcolor[rgb]{0.77,0.63,0.00}{#1}}
\newcommand{\RegionMarkerTok}[1]{#1}
\newcommand{\InformationTok}[1]{\textcolor[rgb]{0.56,0.35,0.01}{\textbf{\textit{#1}}}}
\newcommand{\WarningTok}[1]{\textcolor[rgb]{0.56,0.35,0.01}{\textbf{\textit{#1}}}}
\newcommand{\AlertTok}[1]{\textcolor[rgb]{0.94,0.16,0.16}{#1}}
\newcommand{\ErrorTok}[1]{\textcolor[rgb]{0.64,0.00,0.00}{\textbf{#1}}}
\newcommand{\NormalTok}[1]{#1}
\usepackage{graphicx,grffile}
\makeatletter
\def\maxwidth{\ifdim\Gin@nat@width>\linewidth\linewidth\else\Gin@nat@width\fi}
\def\maxheight{\ifdim\Gin@nat@height>\textheight\textheight\else\Gin@nat@height\fi}
\makeatother
% Scale images if necessary, so that they will not overflow the page
% margins by default, and it is still possible to overwrite the defaults
% using explicit options in \includegraphics[width, height, ...]{}
\setkeys{Gin}{width=\maxwidth,height=\maxheight,keepaspectratio}
\IfFileExists{parskip.sty}{%
\usepackage{parskip}
}{% else
\setlength{\parindent}{0pt}
\setlength{\parskip}{6pt plus 2pt minus 1pt}
}
\setlength{\emergencystretch}{3em}  % prevent overfull lines
\providecommand{\tightlist}{%
  \setlength{\itemsep}{0pt}\setlength{\parskip}{0pt}}
\setcounter{secnumdepth}{0}
% Redefines (sub)paragraphs to behave more like sections
\ifx\paragraph\undefined\else
\let\oldparagraph\paragraph
\renewcommand{\paragraph}[1]{\oldparagraph{#1}\mbox{}}
\fi
\ifx\subparagraph\undefined\else
\let\oldsubparagraph\subparagraph
\renewcommand{\subparagraph}[1]{\oldsubparagraph{#1}\mbox{}}
\fi

%%% Use protect on footnotes to avoid problems with footnotes in titles
\let\rmarkdownfootnote\footnote%
\def\footnote{\protect\rmarkdownfootnote}

%%% Change title format to be more compact
\usepackage{titling}

% Create subtitle command for use in maketitle
\newcommand{\subtitle}[1]{
  \posttitle{
    \begin{center}\large#1\end{center}
    }
}

\setlength{\droptitle}{-2em}

  \title{Partiel 2016}
    \pretitle{\vspace{\droptitle}\centering\huge}
  \posttitle{\par}
    \author{Matisse Landais}
    \preauthor{\centering\large\emph}
  \postauthor{\par}
      \predate{\centering\large\emph}
  \postdate{\par}
    \date{21 octobre 2018}


\begin{document}
\maketitle

On veut ne remplacer que 10\% des transistors. Il faut donc commencer
par déterminer la loi que pourrait suivre ces transistors, connaître le
paramètre de cette loi puis déterminer la garantie (un certain t) pour
ne changer que 10\% des transistors.

\begin{Shaded}
\begin{Highlighting}[]
\NormalTok{dureeVie <-}\StringTok{ }\KeywordTok{c}\NormalTok{(}\FloatTok{0.22}\NormalTok{,}\FloatTok{0.24}\NormalTok{,}\FloatTok{0.29}\NormalTok{,}\FloatTok{0.29}\NormalTok{,}\FloatTok{0.33}\NormalTok{,}\FloatTok{0.47}\NormalTok{,}\FloatTok{0.85}\NormalTok{,}\FloatTok{1.14}\NormalTok{,}\FloatTok{1.50}\NormalTok{,}\FloatTok{1.51}\NormalTok{,}\FloatTok{1.64}\NormalTok{,}\FloatTok{1.96}\NormalTok{,}\FloatTok{2.27}\NormalTok{,}\FloatTok{2.44}\NormalTok{,}\FloatTok{2.75}\NormalTok{,}\FloatTok{2.99}\NormalTok{,}\FloatTok{3.15}\NormalTok{,}\FloatTok{3.85}\NormalTok{,}\FloatTok{4.66}\NormalTok{,}\FloatTok{5.04}\NormalTok{,}\FloatTok{5.06}\NormalTok{,}\FloatTok{6.41}\NormalTok{,}\FloatTok{7.58}\NormalTok{,}\FloatTok{7.81}\NormalTok{,}\FloatTok{8.00}\NormalTok{,}\FloatTok{8.24}\NormalTok{,}\FloatTok{10.15}\NormalTok{,}\FloatTok{12.24}\NormalTok{,}\FloatTok{13.78}\NormalTok{,}\FloatTok{16.12}\NormalTok{)}
\end{Highlighting}
\end{Shaded}

On peut commencer par tracer les histogrammes (classes de même effectif)

\begin{Shaded}
\begin{Highlighting}[]
\NormalTok{histoeff <-}\StringTok{ }\ControlFlowTok{function}\NormalTok{(x, }\DataTypeTok{xlim=}\OtherTok{NULL}\NormalTok{, ...)}
\NormalTok{\{}
\NormalTok{  sx <-}\StringTok{ }\KeywordTok{sort}\NormalTok{(x)}
\NormalTok{  n <-}\StringTok{ }\KeywordTok{length}\NormalTok{(x)}
\NormalTok{  k <-}\StringTok{ }\KeywordTok{round}\NormalTok{(}\KeywordTok{log}\NormalTok{(n)}\OperatorTok{/}\KeywordTok{log}\NormalTok{(}\DecValTok{2}\NormalTok{)}\OperatorTok{+}\DecValTok{1}\NormalTok{)}
\NormalTok{  rangex <-}\StringTok{ }\KeywordTok{max}\NormalTok{(x)}\OperatorTok{-}\KeywordTok{min}\NormalTok{(x)}
\NormalTok{  quantileVoulu <-}\StringTok{ }\KeywordTok{quantile}\NormalTok{(x, }\KeywordTok{seq}\NormalTok{(}\DecValTok{1}\NormalTok{,k}\OperatorTok{-}\DecValTok{1}\NormalTok{)}\OperatorTok{/}\NormalTok{k, }\KeywordTok{max}\NormalTok{(x))}

\NormalTok{  breaks <-}\StringTok{ }\KeywordTok{c}\NormalTok{(}\KeywordTok{min}\NormalTok{(x)}\OperatorTok{-}\FloatTok{0.025}\OperatorTok{*}\NormalTok{rangex, quantileVoulu, }\KeywordTok{max}\NormalTok{(x)}\OperatorTok{+}\FloatTok{0.025}\OperatorTok{*}\NormalTok{rangex)}
\NormalTok{  col <-}\StringTok{ }\DecValTok{0}
  \ControlFlowTok{if}\NormalTok{ (}\KeywordTok{is.null}\NormalTok{(xlim)) xlim<-}\KeywordTok{c}\NormalTok{(breaks[}\DecValTok{1}\NormalTok{], breaks[k}\OperatorTok{+}\DecValTok{1}\NormalTok{])}
\NormalTok{  colors =}\StringTok{ }\KeywordTok{c}\NormalTok{(}\StringTok{"red"}\NormalTok{, }\StringTok{"yellow"}\NormalTok{, }\StringTok{"green"}\NormalTok{, }\StringTok{"violet"}\NormalTok{, }\StringTok{"orange"}\NormalTok{, }\StringTok{"blue"}\NormalTok{, }\StringTok{"pink"}\NormalTok{, }\StringTok{"cyan"}\NormalTok{)}
  
  \KeywordTok{hist}\NormalTok{(x, }\DataTypeTok{breaks=}\NormalTok{breaks, }\DataTypeTok{col=}\NormalTok{colors, }\DataTypeTok{xlim=}\NormalTok{xlim, }\DataTypeTok{probability=}\NormalTok{T)}
\NormalTok{\}}

\KeywordTok{histoeff}\NormalTok{(dureeVie)}
\end{Highlighting}
\end{Shaded}

\includegraphics{Partiel_2016_files/figure-latex/unnamed-chunk-2-1.pdf}

On peut observer que cela ressemble à une loi exponentielle. En traçant
le graphe de probabilités pour la loi exponentielle on peut se
rapprocher de cette hypothèse.

\begin{Shaded}
\begin{Highlighting}[]
\NormalTok{graphProba <-}\StringTok{ }\ControlFlowTok{function}\NormalTok{(xi,hxi,limy)}
\NormalTok{\{}
  \KeywordTok{plot}\NormalTok{(xi,hxi)}
  \KeywordTok{abline}\NormalTok{(}\DataTypeTok{v=}\DecValTok{0}\NormalTok{)}
  \KeywordTok{abline}\NormalTok{(}\DataTypeTok{h=}\DecValTok{0}\NormalTok{)}
\NormalTok{\}}

\CommentTok{# x représente le tableau non trié avec les observations sur les ampoules}
\CommentTok{# log(1 - seq(1:9)/10) est la transformation de la fonction de répartition de la loi exponentielle}
\CommentTok{# sort(x) [1:9] est le tableau x trié en ne gardant que les 9 premiers éléments}
\NormalTok{lastElem <-}\StringTok{ }\KeywordTok{length}\NormalTok{(dureeVie)}
\NormalTok{blElem <-}\StringTok{ }\KeywordTok{length}\NormalTok{(dureeVie)}\OperatorTok{-}\DecValTok{1}
\CommentTok{#limy<-c(-2.5,0.1)}
\NormalTok{myxi <-}\StringTok{ }\KeywordTok{sort}\NormalTok{(dureeVie)[}\DecValTok{1}\OperatorTok{:}\NormalTok{blElem]}
\NormalTok{myhxi <-}\StringTok{ }\KeywordTok{log}\NormalTok{(}\DecValTok{1}\OperatorTok{-}\KeywordTok{seq}\NormalTok{(}\DecValTok{1}\OperatorTok{:}\NormalTok{blElem)}\OperatorTok{/}\NormalTok{lastElem)}


\KeywordTok{graphProba}\NormalTok{(myxi,myhxi)}
\end{Highlighting}
\end{Shaded}

\includegraphics{Partiel_2016_files/figure-latex/unnamed-chunk-3-1.pdf}

Je pense que la loi est une loi exponentielle et j'essaie de déterminer
une approximation du paramètre de cette loi, \(\lambda\). On utilise le
maximum de vraisemblance pour l'estimer :\\
\[\widehat{\lambda} = \frac{1}{\overline{X_n}}\] ou \(X_n\) est la
moyenne de mon échantillon.

\begin{Shaded}
\begin{Highlighting}[]
\NormalTok{moyenneEmpirique <-}\StringTok{ }\KeywordTok{mean}\NormalTok{(dureeVie)}
\NormalTok{lambdaApp =}\StringTok{ }\DecValTok{1}\OperatorTok{/}\NormalTok{moyenneEmpirique}
\KeywordTok{print}\NormalTok{(lambdaApp)}
\end{Highlighting}
\end{Shaded}

\begin{verbatim}
## [1] 0.2255978
\end{verbatim}

\begin{Shaded}
\begin{Highlighting}[]
\NormalTok{chip <-}\StringTok{ }\ControlFlowTok{function}\NormalTok{(x) lambdaApp }\OperatorTok{*}\StringTok{ }\KeywordTok{exp}\NormalTok{(}\OperatorTok{-}\NormalTok{lambdaApp }\OperatorTok{*}\StringTok{ }\NormalTok{x)}
\KeywordTok{curve}\NormalTok{(chip, }\DataTypeTok{from=}\DecValTok{0}\NormalTok{, }\DataTypeTok{to=}\DecValTok{18}\NormalTok{)}
\end{Highlighting}
\end{Shaded}

\includegraphics{Partiel_2016_files/figure-latex/unnamed-chunk-4-1.pdf}

Calculons maintenant le biais de l'estimateur. On peut tirer 1000000
valeurs avec notre estimateur et voir de combien s'eloigne la moyenne de
ces valeurs de la moyenne de notre echantillon.

\begin{Shaded}
\begin{Highlighting}[]
\KeywordTok{print}\NormalTok{(}\KeywordTok{lm}\NormalTok{(myxi}\OperatorTok{~}\NormalTok{myhxi))}
\end{Highlighting}
\end{Shaded}

\begin{verbatim}
## 
## Call:
## lm(formula = myxi ~ myhxi)
## 
## Coefficients:
## (Intercept)        myhxi  
##     -0.1614      -4.4394
\end{verbatim}

\begin{Shaded}
\begin{Highlighting}[]
\NormalTok{vals <-}\StringTok{ }\KeywordTok{rexp}\NormalTok{(}\DecValTok{10000000}\NormalTok{,lambdaApp)}
\NormalTok{diffMean <-}\StringTok{ }\KeywordTok{abs}\NormalTok{(}\KeywordTok{mean}\NormalTok{(vals) }\OperatorTok{-}\StringTok{ }\NormalTok{moyenneEmpirique)}
\KeywordTok{print}\NormalTok{(diffMean)}
\end{Highlighting}
\end{Shaded}

\begin{verbatim}
## [1] 0.00287722
\end{verbatim}

NB : la moyenne est bonne, voyons maintenant l'écart type et la variance

Dire que l'on veut ne remplacer que 10\% des transistors si la garantie
est de t nombre d'années, cela revient à dire que la probabilité qu'on
tire un transistors dont la durée de vie sera inférieur à t est de 10\%.
Avec la fonction de répartition, cela correspond à :
\[F_x(t) = P(X \leq t) = 0.1\]

Il faut déterminer t. Comme \(F_x(t)\) est strictement croissante, elle
est inversible. Alors : \[
\begin{align}
t &= F^{-1}(0.1) \\
1 - e^{-\lambda t} &= 0.1 \newline
e^{-\lambda t} &= 0.9 \newline
ln(e^{-\lambda t }) &= ln(0.9) \newline 
-\lambda t &= ln(0.9) \newline
t &= -\frac{ln(0.9)}{\lambda} = F^{-1}(0.1)
\end{align}
\]

\begin{Shaded}
\begin{Highlighting}[]
\NormalTok{t <-}\StringTok{ }\NormalTok{(}\OperatorTok{-}\DecValTok{1}\OperatorTok{*}\KeywordTok{log}\NormalTok{(}\FloatTok{0.9}\NormalTok{))}\OperatorTok{/}\NormalTok{lambdaApp}
\CommentTok{# En milliers d'heures}
\KeywordTok{print}\NormalTok{(t)}
\end{Highlighting}
\end{Shaded}

\begin{verbatim}
## [1] 0.467028
\end{verbatim}

\begin{Shaded}
\begin{Highlighting}[]
\CommentTok{# En heures}
\KeywordTok{print}\NormalTok{(t}\OperatorTok{*}\DecValTok{1000}\NormalTok{)}
\end{Highlighting}
\end{Shaded}

\begin{verbatim}
## [1] 467.028
\end{verbatim}

\begin{Shaded}
\begin{Highlighting}[]
\CommentTok{# En jours}
\KeywordTok{print}\NormalTok{((t}\OperatorTok{*}\DecValTok{1000}\NormalTok{)}\OperatorTok{/}\DecValTok{24}\NormalTok{)}
\end{Highlighting}
\end{Shaded}

\begin{verbatim}
## [1] 19.4595
\end{verbatim}

Il faudrait fixer la garantie à 0.467 milliers d'heures pour n'avoir que
10\% des transistors à changer.


\end{document}
